\documentclass[a4paper]{article}

\usepackage{amsmath}
\usepackage{amsfonts}
\usepackage{graphicx}
\usepackage{enumerate}
\usepackage{hyperref}
\usepackage{anysize}
\usepackage{hyperref}
\usepackage{listings}
\usepackage{color}

\marginsize{2.5cm}{2.5cm}{1.5cm}{1.5cm}
\setlength{\parindent}{0pt}

\graphicspath{{./images/}}

\title{P\&D Encryption---Final Report}
\author{Pieter Maene and Stijn Meul}
\date{\today}

\begin{document}
\lstset{
  language=C,                % choose the language of the code
  numbers=left,                   % where to put the line-numbers
  stepnumber=1,                   % the step between two line-numbers.        
  numbersep=10pt,                  % how far the line-numbers are from the code
  backgroundcolor=\color{white},  % choose the background color. You must add \usepackage{color}
  showspaces=false,               % show spaces adding particular underscores
  showstringspaces=false,         % underline spaces within strings
  showtabs=false,                 % show tabs within strings adding particular underscores
  tabsize=2,                      % sets default tabsize to 2 spaces
  captionpos=b,                   % sets the caption-position to bottom
  breaklines=true,                % sets automatic line breaking
  xleftmargin=30pt,				  % set the margin from the left
  basicstyle=\footnotesize\ttfamily, 
  breakatwhitespace=true,         % sets if automatic breaks should only happen at \lstinputlisting;
}

\maketitle

\section{Overview}

\section{Application Description}

The application consists of two main parts: the handshake and the encryption of data packets. For the handshake we use the STS protocol, while the encryption is based on AES-CTR. In this section, we will cover both of them in depth.

\subsection{Handshake}

\subsubsection{Protocol Messages}
In the Station-to-Station protocol, three messages will be exchanged between the sender and receiver. After this three-way handshake both DSPs will have enough information to calculate two keys (a 128-bit AES key, an 80-bit HMAC key) and a 64-bit nonce used in the CTR mode.

\begin{enumerate}
    \item The sender generates a random number $x$ and then sends $\alpha^x\ mod\ p$ to the receiver. This is shown in Table~\ref{tab:key_exchange_packet_senderhello}.
    \item The receiver generates a random number $y$ and sends $\alpha^y\ mod\ p$ to the sender. We find the signature by concatenating both exponentials and calculating the HMAC using the derived key. Next the RSA signature of this data is calculated. Finally, this signature is then encrypted using the derived AES key and nonce. This is shown in Table~\ref{tab:key_exchange_packet_receiverhello}.
    \item The sender acknowledges the message from the receiver by sending its signature to the receiver. To obtain this, we first concatenate both exponentials and calculate the HMAC using the derived key. Next the RSA signature of this data is calculated. This is shown in Table~\ref{tab:key_exchange_packet_senderacknowledge}. Here too, we encrypt the signature using the derived key and CTR nonce. This packet is fundamental to the protocol, as both the sender an receiver can use it to verify whether both have all the correct data.
\end{enumerate}

\begin{table}[H]
    \begin{center}
        \begin{tabular}{| c | c | c | c | c |}
            \hline
            TAG & KEY \\ \hline
            1B & 156B \\ \hline
            0x00 & $\alpha^x\mod{p}$ \\
            \hline
        \end{tabular}
    \end{center}
    \
    \caption{Key Exchange Packet---SenderHello}
    \label{tab:key_exchange_packet_senderhello}
\end{table}
\begin{table}[H]
    \begin{center}
        \begin{tabular}{| c | c | c | c | c |}
            \hline
            TAG & KEY & SIGNATURE \\ \hline
            1B & 156B & 20B \\ \hline
            0x01 & $\alpha^y\mod{p}$ & $\left(\text{SHA2}\left(\text{PKCS}\left(\alpha^y \big| \alpha^x \right)\right)\right)^{d_R}\mod{n_R}$\\
            \hline
        \end{tabular}
    \end{center}
    
    \caption{Key Exchange Packet---ReceiverHello}
    \label{tab:key_exchange_packet_receiverhello}
\end{table}
\begin{table}[H]
    \begin{center}
        \begin{tabular}{| c | c | c | c | c |}
            \hline
            TAG & KEY & SIGNATURE \\ \hline
            1B & 156B & 20B \\ \hline
            0x02 & $\alpha^y\mod{p}$ & $\left(\text{SHA2}\left(\text{PKCS}\left(\alpha^x \big| \alpha^y \right)\right)\right)^{d_S}\mod{n_S}$\\
            \hline
        \end{tabular}
    \end{center}
    
    \caption{Key Exchange Packet---SenderAcknowledge}
    \label{tab:key_exchange_packet_senderacknowledge}
\end{table}

We decided to use SHA2 instead of SHA3 (which we chose in previous reports) because there is no standardised RSA padding scheme (like PKCS for SHA2) available yet for this hash function. We evualated our RSA implementation using a test vector provided by Mikhail Fomichev and Shayan Kaman Zadeh (Crypto 8).

\subsubsection{Signature Calculation}

\subsection{Data Encryption}

\section{Optimizations}
\subsection{Optimization Table}
\begin{center}
	\begin{tabular}{| l | c | r |}
		\hline
		Code Stage & Number of Cycles & Code Gain \\ \hline
		Base Code & $6,20662313 \cdot 10^{8}$ 	& $0\%$ \\
		CRT Optimization & $4,35591090 \cdot 10^{8}$ & $29,82\%$ \\
		USE64WITH32 Optimization 	& $4,29177952 \cdot 10^{8}$ & $1,03\%$ \\
		Optimization of Conversion Functions & $4,24002677 \cdot 10^{8}$ & $0,83\%$ \\
		Restrict Optimization	 & $1,80351722 \cdot 10^{8}$ & $39,26\%$ \\
		\hline
	\end{tabular}
\end{center}
\subsection{CRT Optimization}
	The implementation of the Chinese Remainder Theorem is an algorithmic optimization and not a DSP specific one. 
	\subsubsection{Example}
		SOME THEORETICAL EXAMPLE HERE
	\subsubsection{Impact of the Optimization}
\subsection{USE64WITH32 Optimization}
	USE64\_WITH32 is a $\#$define statement in the BigDigits library we are using. If this switch is activated, 64-bit behavior is simulated in 32-bit environment in C.
	\subsubsection{Example}
		When the USE\_64WITH32 flag is set the following code is called for the multiply function. The performance gain can be seen as there is no seperate code to do the transfer of carry bits from one 32-bit variable to another.
		\lstinputlisting[language=C]{source/use64.c}
	\subsubsection{Impact of the Optimization}
		The impact of this optimization is quite large as the multiply function is called many times during the handshake. This gives an optimization of $1,85071223\cdot10^{8}$ cycles.
\subsection{Optimization of Conversion Functions}
	This optimization is again an algorithmic specific optimization. 
\subsection{Restrict Optimization}
	The addition of the restrict keyword in front of all our pointer variables is a real DSP optimization.
	\subsubsection{Example}
		The only changes that need to be made to implement the restrict optimization is adding the restrict keyword in front of all the pointer arguments in the function headers. An example of a function header with the restric keyword:
		\lstinputlisting[language=C]{source/restrictheader.c}
		An example of a function header before the restrict optimization:
		\lstinputlisting[language=C]{source/norestrictheader.c}
	\subsubsection{Impact of the Optimization}
	When the compiler is compiling code with a lot of pointer arguments, it is being very pessimistic about the content of the pointer arguments. For the compiler it is much safer to suppose all the contents of the pointer are changed during the function call except that in most implementations this assumption is too strong. When adding the restrict keyword in front of pointer arguments, the compiler supposes that only the current function is altering the values of the pointer. By doing so the compiler is able to parallelize most of the copying from and to that variable which gives a great improvement.\\
	
	As the average pointer variable in our code has a length of approximately 150 bytes, this optimization results in a much faster execution time. $2,43650955\cdot10^{8}$ cycles are saved by implementing this DSP optimization.
\section{Porting and Integration}

\section{Lessons Learnt}

\section{Conclusions}

\end{document}
