\documentclass[a4paper]{article}

\usepackage{amsmath}
\usepackage{amsfonts}
\usepackage{graphicx}
\usepackage{enumerate}
\usepackage{hyperref}
\usepackage{anysize}
\usepackage{hyperref}

\marginsize{2.5cm}{2.5cm}{1.5cm}{1.5cm}
\setlength{\parindent}{0pt}

\graphicspath{{./images/}}

\title{P\&D Encryption---Final Report}
\author{Pieter Maene en Stijn Meul}
\date{\today}

\begin{document}
\maketitle

\section{Overview}

\section{Application Description}

\section{Optimizations}
\subsection{Optimization Table}
\begin{center}
	\begin{tabular}{| l | c | r |}
		\hline
		Code stage 								& Number of Cycles 				& Code Gain \\ \hline
		Base Code 								& $6,20662313 \cdot 10^{8}$ 	& $0\%$ 	\\
		CRT Optimization 						& $4,35591090 \cdot 10^{8}$		& $29,82\%$ \\
		USE64WITH32 Optimization 				& $4,29177952 \cdot 10^{8}$		& $1,03\%$	\\
		Optimization of Conversion Functions	& $4,24002677 \cdot 10^{8}$		& $0,83\%$	\\
		Restrict Optimization					& $1,80351722 \cdot 10^{8}$		& $39,26\%$ \\
		\hline
	\end{tabular}
\end{center}
\subsection{CRT Optimization}
	The implementation of the Chinese Remainder Theorem is an algorithmic optimization and not a DSP specific one. 
	\subsubsection{Example}
		SOME THEORETICAL EXAMPLE HERE
	\subsubsection{Impact of the Optimization}
	\subsubsection{Reasoning Behind the Impact}
\subsection{USE64WITH32 Optimization}
	USE64WITH32 is a $\#$define statement in the BigDigits library we are using. If this switch is activated, 64-bit behavior is simulated in 32-bit environment.
	\subsubsection{Example}
		PASTE SOME RAW ASSEMBLY CODE HERE
	\subsubsection{Impact of the Optimization}
	\subsubsection{Reasoning Behind the Impact}
\subsection{Optimization of Conversion Functions}
	This optimization is again an algorithmic specific optimization. 
\section{Porting and Integration}

\section{Lessons Learnt}

\section{Conclusions}

\end{document}
